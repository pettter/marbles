\section{Project plan and method}

The project was conceptually divided into five phases, with several weeks
devoted to each.

\subsection{Planning}

The first phase of the project consisted of planning the work ahead, as
well as reading a lot of literature in various areas to get an idea of what
that work would consist of, more specifically. This was supposed to be a
rather small part of the overall project, taking no more than a week or two
to complete.

Project ''deliverables'' as it were, included an outline of a project plan,
as well as a general idea of the subject.

\subsection{Design}

Designing the basic architecture would require even more reading, and
deciding upon the actual programming language to use (as well as learning
that language, if required). As such, it was expected that this would take
a lot more time to do than the planning; about 3-4 weeks.

Also included in this phase would be to further investigate previous work
in the area of tree automata and general FSM programming frameworks.

\subsection{Implementation}

As for the basic implementation of the framework, this was expected to be
one of the major parts of the project, taking at least five weeks to
complete somewhat. Note that this was supposed to be "just" the internals
of the framework, with GUI and more concrete and "usable" classes being
worked on in the next phase.

Still, this phase was to result in a minimally useful framework that could
be used in further coding.

\subsection{Concretisation}

With a completed "framework", concretisation of the GUI, several other
useful classes, and the various Grammatical Inference algorithms was to
commence. Also, quite a bit of testing was to be done between the various
algorithms, to determine various performace characteristics, and how these
related to each other.

\subsection{Writing}

Finally, this report was to be written, along with the analysis of the test
results. This was also expected to take quite a bit of time, and about six
weeks were devoted to this phase in the original plan.
