\section{Obstacles}

As expected, things did not go according to plan. Various delays, bugs, and
other hold-ups caused much of the planned work to be omitted from the final
result. Although certain of these problems may be attributed to external
factors, mostly they were the result of either coding mistakes,
misunderstandings of the standard library of Scala 2.8, or lack of
motivation ("writer's block").

\subsection{Code}

The first major hold-up in the schedule was the realisation that at least
the internals of the framework should be written in Scala, mostly thanks to
the incredibly powerful typing system, but also owing a lot to its
foundation as a functional language.

Thus, learning to program in Scala, using Martin Oderskys excellent book,
took about a week to get the basics down, and another week to get
proficient enough to dare do some proper coding (as well as realising what
version to use, the soon-to-be outdated 2.7 branch, or the
still-in-progress 2.8 beta). With this done, however, the rest of the
coding, especially the later parts, took far less time than they otherwise
would have.

However, there were still quite a few problems encountered that kept
progress from being smooth. Most notably, an inconsistency in the standard
library resulted in strange typing errors in several places. After a few
days of digging, the problem was pointed out in the official scala IRC
channel, and fixed upstream.

A further problem arose when the basics were done, and more advanced
algorithms and structures were to be implemented. First of all, most
algorithms are described in an imperative manner in literature, and
translating these to functional languages is not the easiest exercise at
all times. Second, for rapid testing, it is preferrable to be able to
describe trees and other objects in a more natural way than is possible
using nested classes. Compare the two strings in fig %TODO
for an example. Thus, some kind of parsing should be included in the basic
classes. Luckily, the standard library of Scala includes exactly such a
parser, in the form of the \texttt{scala.util.parsing.combinator} library.
It is not trivially useful, though, and learning the proper ways to utilise
this powerful utility took quite some time.

\subsection{Motivation}

More insidious, then, were the problems of motivation. This was especially
apparent during the final month and a half, when colleagues and contacts on
both hemispheres in general were having their vacations, as well as the
work entering a phase where it was apparent that many of the original
project goals were not going to be reached. Further compounding the problem
was the fact that much of the off-time activities were limited to and by
the ongoing football world cup. Once the duration in south africa was
almost up, physical injury put a further dent in mental fortitude in the
form of a broken foot.

