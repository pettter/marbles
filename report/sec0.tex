\section{Introduction}

%XXX This might be more of an abstract...
In \cite{drewes_marbles}, Drewes outlines Marbles, a programming framework
for working in a generic and systematic way, not only on trees, as several
frameworks already exist for this purpose, but on tree recognisers,
transducers, generators and other formal devices as well (collectively:
tree automata). This thesis focuses on a proposed implementation of this
framework, with special focus on the choice of language and ease of
extension.

Formal tree language theory is an active area of research that is
continually finding new applications within such diverse subjects as model
checking, compiler design and natural language processing. 

The theoretical basis of formal tree language theory is the regular string
automata theory popularised and expanded by Chomsky, Shützenberger et. al.
in the 1950's. Specifically, the automata hierarchies of the string cases
have rough equivalents when working on trees, and many results that hold
for the string case are applicable to the tree case. 

Trees and tree automata has been an area of finite-automata theoretical
research since the 1960's, when researchers realised that large parts of
traditional string automata theory could be extended by replacing strings
with trees. Through finding ever more applications in recent years, the
area sees more research than ever before. 

Implementation of various tree automata and the algorithms working on same
have been fragmented, however, and data and code exchange has suffered as a
result. Several efforts are nevertheless worthy of mention, not least the
Treebag\cite{drewes_treebag} system, as Marbles is in a large part based on
ideas from that project.


This section will continue with a general introduction to the subject
of the thesis, including a discussion on previous work in the area and an
introduction to the goals of the thesis project. Section
\ref{sec_preliminaries} will dig deeper into the theoretical fundaments
required for the rest of the thesis, while Section \ref{sec_method} will
describe the method used to reach the project goals. Section
\ref{sec_method} also includes a discussion on the choice of programming
language, as well as the basic architecture of the prototype system.

Section \ref{sec_results} will discuss the resulting system, and some of
the implications of the choices made in the prototyping process. In
particular, how the implementation of certain algorithms differ from their
generic description. Further, several proof-of-concept algorithm
implementations will be presented and discussed.

Finally, Section \ref{sec_conclusions} will contain a number of closing
remarks, and introduce the next steps in making the prototype into an
actual working framework, useable for research purposes.

\subsection{Motivation and previous work}


%TODO: previous work

\subsection{Project goals}

While the above projects and systems are very useful in their specific
domains, they are nevertheless developed to explore those domains, and in
some sense constrained by them. Marbles, in contrast, aims to be a
jack-of-all-trades programming framework, supporting exploration of tree
automata and tree-based algorithms through being extensible and flexible
enough for first-approximation work in practically any domain.

As may be apparent, a complete framework of this size and complexity is not
a viable goal for a thesis at the MSc level. Instead, we aim to propose a
viable basis for further research into a complete framework. Specifically,
we aim to
\begin{compactitem}
\item Explore the potential programming language choices
\item Choosing a reasonable subset of tree formalisms for implementation
\item Make a viable prototype for the Marbles system, in particular the
prototype should,
\subitem be usable for at least a small part of the tasks covered by the
full system
\subitem include conrete implementations of some of the concepts described in
\cite{drewes_marbles}, and
\subitem have a reasonable architecture, suitable for further
implementation work, with a view to eventually be expanded into the
complete framework.
\end{compactitem}







This thesis aims to propose an implementation for the ideas outlined in
\cite{drewes_marbles}.  




The use of various tree formalisms has penetrated many different areas of
research. As such, the investigation of trees and tree automata is ever
more relevant and interesting. 




There are many areas of research where 

While working on various topics, researchers frequently encounter
situations which require working on structured data in some way. It is
often useful to structure the data as a tree, where the values of the
subtrees in some way contribute to the values of the whole tree (e.g. XML
data). In the study of formal languages, many algorithms and formal devices
have been developed for handling and defining groups (languages) of trees.



A common problem when working with trees and tree automata is the lack of
generic programming tools, leading to the need for much duplication of
functionality. Contrast this to the many various generic string handling
libraries available, or even included in the standard libraries of various
programming languages.

Though establishing a generic and versatile tree handling is not the focus
of this thesis, it is nevertheless a prerequisite for the end goal. 

That end goal
