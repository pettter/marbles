%XXX This might be more of an abstract...




\section{Introduction}

Techniques based on trees and various tree formalisms have seen increasing
use in many areas in recent years. Perhaps most well-known is the
propensity for using XML as a data exchange medium, but tree-based
techniques have found its place in many other areas, such as natural
language processing, model checking, and compiler optimisation.

Tree techniques, and specifically tree automata techniques are largely
based in the work done by Chomsky et. al. in exploring various string
formalisms and their respective restrictions, such as the Chomsky
hierarchy. During the 1960's, reserarchers started exploring whether the
results obtained in string automata theory and formal languages could
somehow be extended to trees. As the case was, they could, and since then,
tree automata research has been an ever growing area of research, though
practical applications of this research is much more recent.

\subsection{Motivation}

Part of the reason for the lack of application may be attributed to poor
programming language support. While regular expressions have been an
integral part of the UNIX system through command line utilities such as
\texttt{ed} and \texttt{sed}, and later through programming languages such
as Perl, comparable tree techniques have been mostly absent or bulky to
use, both in programming and end user applications.

This lack of programming support has also had a noticeable impact on the
research. String data exchange has been fairly standardised ever since the
ASCII standard was formulated in TODO (with some notable exceptions such as
EBCDIC), which in turn is superceded by Unicode. The only comparable
standard seems to be XML, which was only finally incorporated as a standard
in TODO. XML does not solve the problem of programming language support for
working on trees, however, and various research groups have in many cases
been left to implement the basics of their own tools separately. 

The focus of this thesis is a proposed solution to the identified problem.
The Tree Automata Workbench Marbles, introduced by Drewes in \cite{TODO},
is intended to be an extensible and generic programming framework for
working on not only trees, but tree automata of various types.
Specifically, three main groups of tree automata can be identified:

\begin{compactitem}
\item generators, that produce trees in some well-specified way (usually a
grammar of some sort),
\item recognisers, formal devices that takes a tree and produces some
resulting value. These can be a boolean truth value on if the tree belongs
to a certain language or not, but could in theory be almost anything.
\item transducers, which use a tree as the input, and produces a (usually
different) tree as output.
\end{compactitem}

\subsection{Previous work}

Of course, several projects exist which allow for working on trees and tree
automata in various capacities, though most, as mentioned, are somewhat
narrow in scope, or are not presented in a more organised fashion, but
rather used informally within the confines of a research group.

Certain tools have seen more wide distribution, though, and some of these
have served as an inspiration for the ideas behind Marbles.

\begin{itemize}
\item \textbf{Treebag} is a workbench for working with tree generators and
transducers, stepping through derivations and transductions and inspecting
the effects on the tree. It is further possible to write algebras that work
on the trees to produce some output. The specific context for which Treebag
was developed was tree-based picture generation, but it is useable for many
other tasks relating to trees and tree generators as well. Treebag is in
many ways an ancestor to Marbles, having the same originator and the same
ambition of generality.
\item \textbf{ForestFire} is a toolkit for pattern-matching and tree
acceptance problems, with various algorithms organised into taxonomies. 
\item \textbf{Tiburon} is a package of algorithms for working on various
kinds of tree automata, notably weighted tree transducers and regular tree
grammars. Though the focus is on natural language processing and related
problems in machine learning, the algorithms are general enough to
potentially see use in certain other domains as well.
\item \textbf{Timbuk} is a collection of tools for working on reachability
proofs for term rewriting systems. Recent versions (3.0 and onwards)  no
longer include the tree automata manipulation tools that warrants its
inclusion here, but older versions include various algorithms for emptiness
checking, boolean operations such as automata intersection, union,
inversion etc. 
\end{itemize}

\subsection{Marbles}



\subsection{Project goals}

While the above projects and systems are very useful in their specific
domains, they are nevertheless developed to explore those domains, and in
some sense constrained by them. Marbles, in contrast, aims to be a
jack-of-all-trades programming framework, supporting exploration of tree
automata and tree-based algorithms through being extensible and flexible
enough for first-approximation work in practically any domain.

As may be apparent, a complete framework of this size and complexity is not
a viable goal for a thesis at the MSc level. Instead, we aim to propose a
viable basis for further research into a complete framework. Specifically,
we aim to
\begin{compactitem}
\item Explore the potential programming language choices
\item Choosing a reasonable subset of tree formalisms for implementation
\item Make a viable prototype for the Marbles system, in particular the
prototype should,
\subitem be usable for at least a small part of the tasks covered by the
full system
\subitem include conrete implementations of some of the concepts described in
\cite{drewes_marbles}, and
\subitem have a reasonable architecture, suitable for further
implementation work, with a view to eventually be expanded into the
complete framework.
\end{compactitem}


\subsection{Outline}

Section \ref{sec_preliminaries} will dig deeper into the theoretical
fundaments required for the rest of the thesis, while Section
\ref{sec_method} will describe the method used to reach the project goals.
This section will also include a discussion on the choice of programming
language, as well as the basic architecture of the prototype system.

Section \ref{sec_results} will discuss the resulting system, and some of
the implications of the choices made in the prototyping process. In
particular, how the implementation of certain algorithms differ from their
generic description. Further, several proof-of-concept algorithm
implementations will be presented and discussed.

Finally, Section \ref{sec_conclusions} will contain a number of closing
remarks, and introduce the next steps in making the prototype into an
actual working framework, useable for research purposes.





%
%This thesis aims to propose an implementation for the ideas outlined in
%\cite{drewes_marbles}.  
%
%
%
%
%The use of various tree formalisms has penetrated many different areas of
%research. As such, the investigation of trees and tree automata is ever
%more relevant and interesting. 
%
%
%
%
%There are many areas of research where 
%
%While working on various topics, researchers frequently encounter
%situations which require working on structured data in some way. It is
%often useful to structure the data as a tree, where the values of the
%subtrees in some way contribute to the values of the whole tree (e.g. XML
%data). In the study of formal languages, many algorithms and formal devices
%have been developed for handling and defining groups (languages) of trees.
%
%
%
%A common problem when working with trees and tree automata is the lack of
%generic programming tools, leading to the need for much duplication of
%functionality. Contrast this to the many various generic string handling
%libraries available, or even included in the standard libraries of various
%programming languages.
%
%Though establishing a generic and versatile tree handling is not the focus
%of this thesis, it is nevertheless a prerequisite for the end goal. 
%
%That end goal
%Formal tree language theory is an active area of research that is
%continually finding new applications within such diverse subjects as model
%checking, compiler design and natural language processing. 
%
%The theoretical basis of formal tree language theory is the regular string
%automata theory popularised and expanded by Chomsky, Shützenberger et. al.
%in the 1950's. Specifically, the automata hierarchies of the string cases
%have rough equivalents when working on trees, and many results that hold
%for the string case are applicable to the tree case. 
%
%Trees and tree automata has been an area of finite-automata theoretical
%research since the 1960's, when researchers realised that large parts of
%traditional string automata theory could be extended by replacing strings
%with trees. Through finding ever more applications in recent years, the
%area sees more research than ever before. 
%
%Implementation of various tree automata and the algorithms working on same
%have been fragmented, however, and data and code exchange has suffered as a
%result. Several efforts are nevertheless worthy of mention, not least the
%Treebag\cite{drewes_treebag} system, as Marbles is in a large part based on
%ideas from that project.
