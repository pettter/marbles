\section{Method}

\subsection{Choice of language}

As the prototype was intended to function as a base on which the full
system could be built, much thought was spent on the choice of language.
Ideally, the language would be familiar to a large number of researchers,
while having several desirable features, such as platform-independece, easy
extensibility and a powerful typing system. Initially, the languages
considered were C++, Java and Haskell (C# being seen as far too closely
tied to the Microsoft Windows platform). However, as C++ is both hard to
distribute in a platform-independent form, and notoriously
counterintuitive, the deliberations quickly centered on Java or Haskell,
with Java being seen as being unnecessarily strict and full of boilerplate
code, whereas Haskell instead was unfamiliar to most researchers and hard
to extend and distribute in a reasonable manner. After being informed of
the existance of Scala, however, it seemed the ideal choice.

While at the time it was not as well-known as even Haskell, it nevertheless
had an active community, and showed great promise for the future. Further,
it touted easy integration into the Java ecosystem, meaning researchers
interested in using Marbles would likely be able to write their client code
in Java, and use the Scala parts of the framework behind the scenes.
Strong, static typing, but with heavy use of type inference further tipped
the scales, promising to remove much of the boilerplate required in Java
code. Finally, Scala allows for much coding to be done using a functional
programming paradigm, which offers certain other benefits in code
readability.

%TODO blablabla

\subsection{Basics of Scala}

Scala is a functional-object oriented hybrid language with static typing
and a syntax designed to remove boilerplate and increase legibility. It is
designed to run on both the Java JVM and Microsofts .Net infrastructures,
and has a large standard library that handles much of the underlying
complexities in various common tasks. 

Everything in Scala is an object, down to the integral data types
(\texttt{int} and so on), and functions. However, as opposed to Java or C#,
there is no such thing as a static method. Instead, classes have so-called
''companion objects'', being in principle singletons, usually containing
the static methods and constants, as well as various factory methods.
Companion objects and their companion classes each have private access to
the other.

Probably the most significant difference between Scala and C#/Java is
multiple implementation inheritance. That is, while Java classes inherit
only from a single class, but potentially multiple interfaces, each
interface only describes methods that need to be present in the class, no
actual code to make use of these methods. In contrast, Scala traits are
''rich'', in the sense that they can make use of the defined methods to
provide more functionality. For example, by inheriting (mixing in) the
trait \texttt{Ordered[T]} and implementing the single abstract method
\texttt{compare}, all of the comparison operators (\texttt{< > <= >=})
become available, as well as various sorting methods on collections of the
class.

As a practical example, let us examine the \texttt{Tree} class as defined
in the Marbles prototype:

\begin{verbatim}
/** An ordered tree with nodes of a certain type
 */
class Tree[+T] (val root : T, 
		        val subtrees: Seq[Tree[T]]
			) { 
\end{verbatim}

This defines the \texttt{Tree} class as having a class parameter
(\texttt{T}), with subtyping being covariant in that parameter. That is, a
\texttt{Tree[String]} is considered a subtype of \texttt{Tree[AnyRef]},
which is useful for having generic functions that work on any possible tree
in a consistent way, while avoiding having to parameterise those functions.

Further, we define a default constructor, the relevant parameters/instance
variables, and how these relate to the type parameter. As is readily
apparent, these are a root of type \texttt{T}, and a sequence (list) of
subtrees, each having the same type parameter.

\begin{verbatim}
	/** Substitute all occurances of a certain symbol for a tree over the
	 *  same type (or a supertype) returns a tree over the new type
	 */
	def subst[U >: T](sym : U,sub : Tree[U]) : Tree[U] = {
		subst(Map[U,Tree[U]]((sym ,sub)))
	}

	/** Substitute all occurances of certain symbols for designated 
	 *  trees over the same type (or a supertype) returns a tree over the
	 *  new type
	 */
	def subst[U >: T](subs : Map[U,Tree[U]]) : Tree[U] = {
		if(subs isDefinedAt root)
			subs(root)
		else
			new Tree[U](root,subtrees map(_.subst(subs)) toList)
	}
\end{verbatim}

%TODO: SHould this be straight symbol substitution?

These functions allow us to apply a simple substitution to the tree, that
is, in this tree and all subtrees, we can replace any instance of a
specific symbol by a tree, erasing the previous subtree rooted at that
node. Alternatively, we can do several of these at once, using the second
form. Finally, and most importantly, while we cannot change the type of the
resulting tree to \emph{any} type, as long as the symbol type is a subclass
of the substitute class (the \texttt{[U >: T]} restriction on the type
parameter), the transformation still makes sense. 

For example, if the classes \texttt{Rectangle} and \texttt{Circle} both
inherit from the class \texttt{Shape}, then a \texttt{Tree[Rectangle]}
could have a susbtitution applied that changed, say, the first ten
integer-sized squares to \texttt{Circles} instead, resulting in a
\texttt{Tree[Shape]}.

\begin{verbatim}	
	/** Apply a function to every node of this tree.
	 */
	def map[To](f : (T) => (To)):Tree[To] = {
		new Tree(f(root), subtrees map (_ map f))
	}
\end{verbatim}

While the substitution transformation is not required to substitute every
possible symbol for another (indeed, they will often just substitute a
few), the \texttt{map} method instead applies a function to each node in
the tree, and returns the result. In many cases, the resulting type
(\texttt{To}) will have no particular relation to the current tree type.

\begin{verbatim}
	
	/** Get the leaf trees of this tree
	 */
	def leaves:Seq[Tree[T]] = if(subtrees.size == 0)
										this :: Nil 
									else
										subtrees flatMap (_.leaves)
\end{verbatim}

This returns a left-to-right list of the leaves of the tree, in effect
applying the \texttt{yield} function to the tree. Specific Scala constructs
of note are using the cons operator (\texttt{::}) to create a list, and
the \texttt{flatMap} function to first apply the function to each member of
\texttt{subtrees} and then flattening the resulting lists
(\texttt{[["a","b"],["c"]]} becomes \texttt{["a","b","c"]})

\begin{verbatim}
	override def toString : String = subtrees.size match { 
		case 0 => root toString
		case _   => root.toString + subtrees.mkString("[",",","]") 
	}

	override def equals(other : Any) = other match{
		case that : Tree[_] => this.root == that.root &&
							   this.subtrees == that.subtrees
		case _ => false
	}

	override def hashCode = 41 * ( 41 + root.hashCode) + subtrees.hashCode
}
\end{verbatim}

Finally, we override some functions to make printing, comparisons and
hashing possible. Without these, the standard \texttt{AnyRef} methods would
be called instead, but these are based on object reference addresses, which
is not suitable for most applications, though they are no doubt faster than
the above methods. 

In addition to the class itself, there is a companion object defined for
the \texttt{Tree} class:
\begin{verbatim}
/** Companion object for the Tree class. Contains factory and constructor
 *  methods, as well as extractors.
 */
object Tree {
	/** Standard factory method
	 */
	def apply[T](rt:T,ss:Seq[Tree[T]]) = new Tree(rt,ss)
	
	/** Factory for leaves
	 */
	def apply[T](rt:T) = new Tree(rt,Nil)
	
\end{verbatim}

Including various factory methods in the companion object is standard
practise for Scala classes. The special method name \texttt{apply} can be
omitted in calls, i.e. \texttt{Tree(foo, bar)} is equivalent to
\texttt{Tree.apply(foo, bar)}. Unsurprisingly, this is how function objects
are defined in Scala. 

\begin{verbatim}
	/** Extracts the root and subtrees of a tree.
	 */
	def unapply[T](t:Tree[T]):(T,Seq[Tree[T]]) = (t.root,t.subtrees)
\end{verbatim}

The special method \texttt{unapply} works as a sort of \texttt{apply} in
reverse. Specifically, it can be used to extract information from arbitrary
classes in pattern matching and similar situations. For example, you can do
pattern matchin on \texttt{Tree}s as such:

\begin{verbatim}
t match {
  case Tree(root:T, Nil) =>  //Do something 
  case Tree(root:U, Nil) =>  //Do something
  case Tree(root:U, subtrees) =>  //Do something
}
\end{verbatim}

Again, this is largely syntactic sugar to make things more intuitive than
would otherwise be possible.

\begin{verbatim}
	/** A parser for trees on the form "root[tree,...]" or "root" if the
	 *  tree is a leaf. 
	 */
	implicit def treeParser[T](implicit rootParsers:ElementParsers[T]) = new ElementParsers[Tree[T]] {
		val root:Parser[T] = rootParsers
		def tree:Parser[Tree[T]] =
			(root~opt("["~>repsep(tree,",")<~"]")) ^^ {
				case root~None       => new Tree(root,Nil)
				case root~Some(subs) => new Tree(root,subs)
			}

		def start = tree
	}
}
\end{verbatim}

Finally, we define a parser for generic trees. This requires several tricks
to work properly, and it is unknown what other language would be able to
accomplish this so concisely. Specifically, the parser is defined
\texttt{implicit}, meaning it can be accessed as an implicit argument to
functions where it is in scope. The \texttt{treeParser} method also takes
such an implicit argument, specifically a parser \texttt{root} of the
symbol type \texttt{T}. The combinator parsing itself is somewhat obvious,
in that it uses the \texttt{root} parser to parse the root of the specific
subtree, and then optionally reads a \texttt{[}, followed by a number of
subtrees separated by \texttt{,}.

What is interesting to note is that the members of the companion object of
type \texttt{T} is in scope to be included as implicit arguments, meaning
that in order to allow any type to be included as symbols in trees, all
that is required is to include an \texttt{ElementParsers} of the type in
question in its companion object, and then apply the parameterised tree
parser. Alternatively, you could explicitly include a parser as an argument
to the tree parser. %TODO: expand?

In fact, the combinator parsers of the Scala standard library is sufficient
for most of these tasks, but the \texttt{ElementParsers} class was required
in order to allow the parsers to be properly utilised. In particular, type
erasure and a few other issues made defining regular \texttt{Parser}s as
implicit and utilising them properly a difficult, if not impossible, task.  

The Marbles prototype includes parsers for \texttt{String}, \texttt{Int} and
\texttt{Double} in the \texttt{marbles.util} package. See
\texttt{Util.scala} for the specifics.

For further details on Scala syntax and programming, the excellent book
\textit{Programming in Scala} by Odersky et. al. is highly recommended. 


