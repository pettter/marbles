\subsection{Tree automata}

As various types of tree automata form the building blocks of Marbles, we
include their description and implementation as part of the work required
to reach the result rather than as a part of the result itself.

As described in section \ref{ssec_fsa}, finite automata are constructs that
in general have a few things in common, notably an alphabet, a state set,
and a rule set. In Marbles, the \texttt{Alphabet} is simply a \texttt{Set}
of some type \texttt{T}, while a \texttt{RankedAlphabet} is a map from some 
\texttt{T} to \texttt{Int}. States could be of any type, conceptually, but
in the current prototype, they are ;\texttt{String}s. The type of the rule
sets obviously vary between different automata types, but in general are
\texttt{Map}s of the approriate type, with either a single tuple or state
as the right-hand side, or a \texttt{Set}, if the automata type is
non-deterministic.

\subsubsection{Recognisers}

Extending string recognisers (FSA) to the tree case entails, as mentioned,
some way of handling branches, with a choice being made as to moving from
the root downwards (top-down) or from the leaves up (bottom-up) during
processsing. Both of these approaches are available in Marbles, using the
\texttt{TDNFTA} and \texttt{BUNFTA} classes, respectively. With more
theoretical rigour:

A \emph{bottom-up non-deterministic finite tree automaton (BUDFTA)} is a
4-tuple $A = (\Sigma, Q, R, F)$ where
\begin{compactitem}
\item $Q$ is a ranked alphabet of states such that $Q = Q_0$
\item $F \subseteq Q$ is a ranked alphabet of final states
\item $\Sigma$ is the (ranked) \emph{tree alphabet} and
\item $R$ is a set of rules on the form
$a[q_1 \ldots q_k] \rightarrow q'$ for $q',q_1 \ldots q_k \in Q, a \in
\Sigma_k$
\end{compactitem}
\vspace{0.4cm}

In Marbles, this is represented by the \texttt{BUNFTA} class, which
contains the instance variables \texttt{sigma}, \texttt{states},
\texttt{rules}, and \texttt{fin}, which obviously corresponds exactly to
the structure described above. The rule set is of the type
\texttt{Map[(T,Seq[String]),Set[String]]}, which, again, corresponds rather
exactly to how we describe them in algorithms. As was mentioned in the
introduction to this section, we use a \texttt{Set} to gather the various
right-hand sides corresponding to a particular left-hand side.

Moving on with the theoretical definition, a configuration of a BUNFTA $A$
running on a tree $t$ is a 4-tuple
$C = (A, t, \Sigma', t')$ where
\begin{compactitem}
\item $A$ is the automaton,
\item $t$ is the tree,
\item $\Sigma'$ is a ranked alphabet with $\Sigma'_k = \Sigma_k$ for $k >
$0 and $\Sigma'_0 = \Sigma_0 \cup Q$, and
\item $t'$ is tree over $\Sigma'$, called the position tree.
\end{compactitem}
\vspace{0.5cm}

A run of a BUNFTA $A$ on a tree $t$ is a sequence of configurations $C_0
\ldots C_l$ such that for all configurations, $A$, $t$ and $\Sigma'$ are
equal, and for each successive pair of configurations $C_n$, $C_{n-1}$, the
trees $t'_n$ and $t'_{n-1}$ are related as follows: 
\begin{compactitem}
\item There is a subtree $a[q_1 \ldots q_k], q_1 \ldots q_k \in Q$ at a position $p$ in $t'_{n-1}$
\item there is a subtree $q'\in Q$ at position $p$ in $t'_n$
\item $t'_{n-1}$ and $t'_n$ are otherwise equal, and
\item there is a rule $a[q_1 \ldots q_k] \rightarrow q'$ in $R$
\end{compactitem}
\vspace{0.5cm}

An accepting run ${C_0 \ldots C_n}$ of a BUNFTA $A$ on a tree $t$ is a run where
\begin{compactitem}
\item in $C_0, t' = t$ and
\item in $C_n, t' \in F$ %TODO: Update other run-definitions
\end{compactitem}
\vspace{0.5cm}

The set $L(A)$ of trees on which an accepting run can be constructed for a
BUNFTA $A$ is the language of the automaton. The class of languages
recognised by BUNFTA is the \emph{regular tree languages}.

In Marbles, running the automaton on a tree to see if it part of the
language is a simple manner of using the \texttt{apply} method, either
explicitly or through just using the \texttt{object(arguments)} syntax.
Further, the \texttt{applyState} method will reveal the exact state set
that a specific subtree ends up in, while \texttt{isDeterministic} checks
if the nondeterminism is actually required for the specific automaton.
Also, parsing of an automaton has been implemented using the combinator
parser.

By restricting the rule set such that each left-hand side appears at most
once, we arrive at the deterministic variant of bottom-up tree automata
(BUDFTA). The expressive power of BUDFTA is exactly equal to BUNFTA, though
the proof of this assertion is outside the scope of this introduction. It
may be briefly mentioned that the proof by construction is similar to the
proof of the equivalence of deterministic and non-deterministic string
automata.

Moving on to the top-down case, a \emph{top-down deterministic finite tree
automaton (TDNFTA)} is a 4-tuple $A = (\Sigma, Q, R, q_0)$ where
\begin{compactitem}
\item $Q$ is a ranked alphabet of states such that $Q = Q_1$
%\item $F \subset Q$ is a ranked alphabet of final state such that $F = F_0$
\item $\Sigma$ is the (ranked) tree alphabet
\item $q_0$ is the initial state and 
\item $R$ is a set of rules on form
$q[a[v_1 \ldots v_k]] \rightarrow q_1[v_1] \ldots  q_k[v_k]$
where $q, q_1 \ldots q_k \in Q, a \in \Sigma_k$, $v_1 \ldots v_n$
are variables,
\end{compactitem}
\vspace{0.5cm}

Again, the Marbles implementation stays close to what is defined in the
theory, with the variables being names \texttt{sigma}, \texttt{state},
\texttt{rules} and \texttt{q0}, respectively, with the rule set being a
\texttt{Map[(T,String),Set[Seq[String]]]}. The one thing to note is that
the right-hand sides are contained in a \texttt{Set} of \texttt{Seq}s. That
is, the state that the automaton uses to traverse downward is dependent on
what state is applied to the sibling trees. The distinction may seem
unimportant, but as will be apparent, it is critical to how the tree
automaton works non-deterministically.

A configuration of a TDNFTA $A$ on a tree $t$ is a 4-tuple $C = (A, t, \Sigma', t')$
where 
\begin{compactitem}
\item $A$ and $t$ are as in BUDFTA configurations,
\item $\Sigma'$ is a ranked alphabet where $\Sigma'_k = \Sigma_k$ for $k \neq 1$ and
$\Sigma'_1 = \Sigma_1 \cup Q$ and
\item $t' \in T_{\Sigma'}$ is a position tree where in the path from the
root to each leaf there is at most one symbol in $Q$.
\end{compactitem}
\vspace{0.5cm}

A run of the TDNFTA $A$ on a tree $t$ is sequence of configurations $C_0
\ldots C_l$ where for each pair $C_n, C_{n-1}$, $A$, $t$, $\Sigma'$ are
equal and for each successive pair $C_n$, $C_{n-1}$, $t'_n$ and $t'_{n-1}$
relate to each other as follows: 
\begin{compactitem}
\item There is a subtree $q[a[t_1 \ldots t_k]]$, $q \in Q$ at position $p$
in $t'_{n-1}$
\item there is a subtree $a[q_1[t_1] \ldots q_k[t_k]]$ at position $p$ in
$t'_{n}$
\item $t'_{n-1}$ and $t'_n$ are otherwise equal, and
\item there is a rule $q[a[v_1 \ldots v_k]] \rightarrow q_1[v_1] \ldots
q_k[v_k]$ in $R$ 
\item as an alternative, the subtrees may be $q[a]$ and $q'$, respectively,
with the rule being $q[a] \rightarrow q$
\end{compactitem}
\vspace{0.5cm}

An accepting run ${C_0 \ldots C_n}$ of a TDNFTA $A$ on a tree $t$ is a run
where in $C_0$, $t' = q_0[t]$, and in $C_n$, $t' \in T_\Sigma$, that is, no
states remain in the intermediate tree.

The set $L(A)$ of trees on which an accepting run can be constructed for the
TDNFTA $A$ is the language accepted by $A$. TDNFTA recognise the regular
tree languages, just as BUNFTA and BUDFTA. Deterministic top-down tree
automata (TDDFTA) can be defined similarly to BUDFTA, that is, we restrict
the rule set such that each left-hand side occurs at most once.

TDDFTA recognise a subclass of the regular tree languages, i.e. there are
regular tree languages for which no TDDFTA can be constructed. An example
of such a language is the language ${f[a,b], f[b,a]}$. To prove this,
assume that there is a rule $$q_0[f[v_1,v_2]] \rightarrow q[v_1], q[v_2]$$
in $R$. This, however, means that in order for both $f[a,b]$ and $f[b,a]$
to be in $L(A)$, there must be rules
$$q[a] \rightarrow $$
$$q[b] \rightarrow $$
in $R$ as well, meaning that both $f[a,a]$ and $f[b,b]$ are in $L(A)$ as
well, meaning that we are recognising the wrong language. It should be
fairly obvious that it is both possible and, in fact, easy to construct a
BUDFTA recognising the correct language.

Further, by allowing non-determinism, we can amend the automaton to have
the rule set
$$q_0[f[v_1,v_2]] \rightarrow q_a[v_1], q_b[v_2]$$
$$q_0[f[v_1,v_2]] \rightarrow q_b[v_1], q_a[v_2]$$
$$q_a[a] \rightarrow $$
$$q_b[b] \rightarrow $$
allowing top-down automata to recognise the same language. Indeed, by
inverting the sides of every rule and switching the initial state set $q_0$
for the final state set $F$, any TDNFTA can be turned into a BUNFTA, and
vice versa, meaning that non-deterministic top-down tree automata also
recognise the class of regular tree languages. This should also clarify why
the non-determinism in Marbles \texttt{TDNFTA} is implemented as a
\texttt{Set} of \texttt{Seq}s instead of the other way around.

\texttt{apply}, \texttt{isDeterministic}, and parsing all work similarly to
how they work for \texttt{BUNFTA}, but \texttt{applyState} returns the
result for processing a specified tree starting in a specified state. 

\subsubsection{Semirings and weighted automata}

A \emph{semiring} is an algebraic structure, used to define \emph{weighted
tree automata (WTA)}, and, using WTA, \emph{recognizable tree series}.
Specifically, a semiring is a set $O$ equipped with two binary operations,
$+$ and $\cdot$ (addition and multiplication), such that
\begin{compactitem}
\item $+$ is a commutative operation on $O$ with identity element $0$
\item $\cdot$ is an operation on $O$ with identity element $1$
\item $\cdot$ distributes over $+$, and
\item multiplication with the additive identity $0$ annihalates $O$, that
is, $a \cdot 0 = 0 \cdot a = 0$
\end{compactitem}

Marbles defines the \texttt{Semiring[T]} and \texttt{SemiringFactory[T]}
traits which may be implemented by the user. Alternatively, one may use one
of the predefined semirings provided in \texttt{semirings.scala}, that is
either
\begin{compactitem}
\item the \texttt{Reals} semiring (basically the real numbers, with $+$, $\cdot$, $0$ and $1$ as would be expected.
\item the \texttt{MaxPlus} semiring, with
\subitem $0 = -\infty$
\subitem $1 = 0$ 
\subitem $+=$ \texttt{max}
\subitem $\cdot=+$
\item the Boolean semiring, with
\subitem $0 =$ \texttt{false}
\subitem $1 =$ \texttt{true}
\subitem $+=$ \texttt{OR}
\subitem $\cdot=$ \texttt{AND}
\end{compactitem}

Informally, a WTA is a function from $T_\Sigma$ to some semiring $O$, using
the multiplication and addition formulas to compute a value from the
subtrees. Using the previous definition of a BUNFTA as a basis, we extend
this to the weighted case as follows:

A \emph{bottom-up non-deterministic weighted tree automaton (BUNWTA)} is a
5-tuple $A = (\Sigma, Q, R, F, O)$ where
\begin{compactitem}
\item $Q$, and $\Sigma$ are as in BUDFTA
\item $F$ is a mapping from $Q$ to $O$ of \emph{final weights}
\item $R$ is a set of rules on the form
$a[q_1 \ldots q_k] \rightarrow q'$ for $q',q_1 \ldots q_k \in Q, a \in
\Sigma_k$
\end{compactitem}
%todo: Write stuff on weighted automata, using Fülöp and Vogler?


\subsubsection{Generators}

The ''inverse'' construct of recognisers are various kinds of
\emph{grammars}. Notable in the string case is the \emph{context-free
  grammar}, which is much more readily used to define context-free
languages than the appropriate recogniser, the push-down automata.
Likewise, the standard regular expression is as easily converted to a
grammar as to a finite string automaton. For trees, the equivalent
construction is the \emph{regular tree grammar}. While recognisers are
reasonable to define in both a top-down and a bottom-up manner, it would be
hard to know in advance how many leaves to start with in a bottom-up
generator. Further, as with the string variants, deterministic grammars are
obviously unreasonable, as such grammars would only define a single string
or tree. Formally:

A \emph{Regular tree grammar (RTG)} RTG is a 4-tuple $G = (\Sigma, N, R, S)$ where
\begin{compactitem}
\item $\Sigma$ is the ranked alphabet of \emph{terminal (output) symbols},
\item $N$ is a ranked alphabet of \emph{non-terminal symbols} such that $N_0 = N$, 
\item $R$ is a set of \emph{rules} on the form\\
    $A \rightarrow t$, where $A \in N$ and $t$ is a tree over $\Sigma \cup N$, and
\item $S \in N$ is the \emph{starting symbol}
\end{compactitem}

An \emph{intermediate tree} $t$ of an RTG $G$ is a tree over $\Sigma \cup N$.

In Marbles, the class \texttt{RTGrammar} is more or less set up as
expected, with the instance variables being named \texttt{sigma},
\texttt{nonterminals}, \texttt{rules} and \texttt{start}, respectively.
Intermediate trees are defined as \texttt{Tree[Either[String,T]]}, and
\texttt{rules} is a \texttt{Map} from \texttt{String} to a \texttt{Set} of
such trees.

A \emph{generation} of a tree $t \in T_\Sigma$ by the RTG $G$ is a sequence $t_0\ldots
t_l$
of intermediate trees of $G$ such that $t_0 = S$, $t_l = t$ and each pair
$t_n, t_{n-1}$ are related to each other and $G$ as follows:
\begin{compactitem}
\item There is a nonterminal $A$ at position $p$ in $t_{n-1}$,
\item there is an intermediate tree $t'$ at position $p$ in $t_n$,
\item $t_{n-1}$ and $t$ are otherwise identical, and
\item there is a rule $A \rightarrow t'$ in $R$.
\end{compactitem}

The \emph{language $L(G)$ generated by the RTG $G$} is the set of trees
that can be generated by $G$. Though the proof is outside the scope of this
thesis, it can be easily shown that RTG correspond exactly to BUNFTA, and
thus is another way to define the regular tree languages.

In Marbles, actual tree generation proved to be much more of a problem than
anticipated. Briefly, there must be an order imposed on the rule set in
order to iterate over it in some meaningful way. However, there is the
additional requirement that no infinite recursion appears. Thus, the rules
were ordered according to how many recursions are required to reach a tree
in $T_\Sigma$. Now, we define a ''rule-choice'' tree, where the root is a
choice of a specific rule among many, and the child trees are the
successive choices of rules among the nonterminals in the intermediate tree
resulting from that rule choice. Thus, iterating over the generated trees
becomes a breadth-first search using these rule-choice trees, and then
applying the rule choices as far as the search has gone. If any
nonterminals remain below the level where the choices have been made, we
choose to use the first rule, since this will terminate the recursion as
fast as possible.

%TODO: Illustrate?

\subsection{Transducers}

Tree transducers are formalised automata that take a tree $t$ as input and
use that to output a tree $t'$. Though not as extensively studied as the
recogniser and generator classes given above, they nevertheless have seen
use in various areas, notably natural language tasks such as translation
(Knight, May). % TODO: others?

As for recognisers, it is reasonable to define both bottom-up and top-down
variants of tree transducers and it will be shown that both variants have
interesting properties. Informally, we can think of tree transducers as
tree recognisers that, apart from producing a state at each node, also
procuce a tree. Additionally, there is a specified way the trees at each
node are combined into one final output tree. More formally:

A \emph{bottom-up finite tree transducer (BUFTT} is a 5-tuple $T = (\Sigma,
\Delta, Q, R, F)$, where
\begin{compactitem}
\item $\Sigma$ is the (ranked) input alphabet,
\item $\Delta$ is the (ranked) output alphabet,
\item $Q$ is a set of states,
\item $R$ is a set of rules on the form
$$s[q_1[x_1],\ldots,q_k[x_k]] \rightarrow q[t']$$ where 
$q,q_1,\ldots,q_k \in Q, s \in \Sigma_k, x_1,\ldots,x_k$ are variables, and
$t' \in T_{\Delta \cup {x_1,\ldots,x_k}}$
\item and $F \subset Q$ is a set of final states.
\end{compactitem}

As for the previous automata types, the Marbles equivalent is fairly close
to the theoretical definition: \texttt{sigma}, \texttt{delta},
\texttt{states} and \texttt{fin} all have the types you would expect, while
\texttt{rules} is of the type \texttt{Map[(F,Seq[String]),Set[(VarTree[T],
  String)]]}, where \texttt{F} is the type paremeter of the input alphabet,
and \texttt{T} of the output alphabet. \texttt{VarTree[T]} is in principle
a \texttt{Tree} over \texttt{Either[Int,T]}, though there are a number of
extra methods implemented for easing the tasks associated with tree
transducers. 

An intermediate tree $t$ of a BUFTT $T$ is a tree over $\Sigma \cup \Delta
\cup Q$, where $Q$ is seen as a ranked alphabet with $Q_1 = Q$. 

A computation of a BUFTT $T$ is a sequence $t_1,\ldots,t_l$ of intermediate
trees such that each pair $t_n, t_{n-1}$ relate to each other as follows:
\begin{compactitem}
\item there exists a tree $s[q_1[t_1],\ldots,q_k[t_k]]$ at position $p$ in
$t_{n-1}$.
\item there exists a tree $q[t'']$ at position $p$ in $t_n$
\item $t_n$ and $t_{n-1}$ is otherwise equal,
\item there is a rule $$s[q_1[x_1],\ldots,q_k[x_k]] \rightarrow q[t']$$ in
$R$, and
\item $t''$ is the tree one obtains by taking $t'$ and substituting each
instance of $x_1$ by $t_1$, each instance of $x_2$ by $t_2$, and so on.
\end{compactitem}

A successful computation of a BUFTT $T$ on a tree $t \in T_\Sigma$ is a
computation $t_1,\ldots,t_l$ where $t_1 = t$ and $t_l = q[t_{out}$ where $q
\in F$ and $t_{out} \in T_\Delta$. The trees $t$, and $t_{out}$ are the
input and output trees, respectively, of this computation. As the BUFTT may
be nondeterminstic, each input tree defines a set of output trees, and the
BUFTT as a whole defines a relation $U$ on $T_\Sigma \times T_\Delta$, where
$(t, t_{out}) \in U$ if there is a successful computation of $T$ such that
$t$ and $t_{out}$ are its input and output trees.

In Marbles, a \texttt{BUTreeTransducer[F,T]} inherits from the trait
\texttt{TreeTransducer[F,T]}, which as of this writing is simply a
''forwarding'' trait that inherits from
\texttt{PartialFunction[Tree[F],Set[Tree[T]]]}. This inserts the transducer
at the appropriate place in the Scala ecosystem, and allows one to use
various interesting constructions, such as making a
\texttt{RegularTreeGrammar}, and then \texttt{map}ping a tree transducer on
top of it, to end up with an \texttt{Iterator} over the sets of output
trees. Alternatively, by using \texttt{flatMap}, the individual trees are
accessed. As for the other automata types, a parser is included in the
companion object.

In a similar way that BUNFTA relate to BUFTT do TDNFTA relate to
\emph{top-down finite tree transducers (TDFTT)}. Formally:

A top-down finite tree transducer (TDFTT) is a 5-tuple $T = (\Sigma,
\Delta, Q, R, q_0)$, where
\begin{compactitem}
\item $\Sigma$ is the (ranked) input alphabet,
\item $\Delta$ is the (ranked) output alphabet,
\item $Q$ is a set of states,
\item $R$ is a set of rules on the form
$$q[s[x_1,\ldots,x_k]] \rightarrow t'$$ where
$q \in Q, s \in \Sigma_k, x_1,\ldots,x_k$ are variables, and
$t' \in T_{\Delta \cup Q \cup {x_1,\ldots,x_k}}$, where $Q$ is seen as a
ranked alphabet with $Q = Q_1$
\item and $q_0 \subset Q$ is a set of initial states.
\end{compactitem}

The Marbles implementation is again fairly close to the theory, but with
\texttt{rules} being of a quite interesting type: \texttt{Map[(F,String),
Set[(VarTree[T], Seq[(String, Int)])]]}. Here, \texttt{(F,String)}
corresponds to the left-hand side quite obviously, but the right-hand side
is more complex: Each \texttt{VarTree} has a number of variables that may
be larger or smaller than the number of subtrees of $s$, so the \texttt{Seq} of
\texttt{(String,Int)} records what state should be used for each particular
variable, and what subtree of $s$ should be inserted at that point.
Obviously, the \texttt{Seq} needs to have the same size as the amount of
variables in the \texttt{VarTree}.

We forego formal definitions of the computations of TDFTT at this time to
focus on what makes TDFTT fundamentally different from BUFTT. In short:
Instead of choosing a tree based on symbol and states from below, and
inserting the subtrees at their respective places, we work from the top,
transforming the tree and nondeterministically choosing the states and
trees as we move downward.  This becomes relevant only when the transducer
is \emph{non-linear}, in the sense that subtrees are copied during the
processing. Specifically, in TDFTT, we can initiate processing of two
copies of the same subtree using two different states, while in BUFTT any
processing will already be complete by the time we are able to apply any
copying. This important distiction means that there are relations that can
be defined by BUFTT but not by TDFTT, and vice versa. This relationship
will be further explored in Section \ref{ssec_tra nsducersplit}

\subsubsection{Weighted transducers}




