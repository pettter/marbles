\section{Results}

In order to demonstrate that the implmemented prototype actually serves the
intended purpose (i.e. as a means to quickly and easily test algorithms on
various tree automata), several algorithms on tree automata were
implemented. During the implementation work, certain implementation
problems seemed to be common for most if not all algorithm implementations. 

\subsection{Functionalising tree automata algorithms}

While it is quite possible to use an imperative programming style even in
Scala, the language is designed to be used in a functional way, and many
aspects of the collections framework among others have plenty of methods
and structures that allow for easy functional programming. This can be
contrasted to many if not most descriptions of algorithms in the
literature, where imperative pseudocode seems prevalent.

\subsection{Example algorithms}

\subsection{Further transducer background}

In order to properly appreciate the example transducer splitting
algorithms, we require some more theory and definitions. Recall the
definitions in Subsection \ref{ssec_transducerdef}. By placing constraints
on the structures, we can find different classes of tree transductions.
Specifically, we call the class of tree transductions definable by TDFTT
$\mathbf{T}$, and by BUFTT, $\mathbf{B}$. Further, by restricting the
number of rules with the same left-hand side to at most one, we arrive at
the \emph{deterministic} TDFTT and BUFTT, respectively, the classes of
transductions definable by these are denoted by $\mathbf{DT}$ and
$\mathbf{DB}$. Note that two left-hand sides are equal if they incorporate
the same symbol and the same state(s), regardless of how the variables are
named.

Additionally, we define the following constraints:
\begin{compactitem}
\item A transducer is \emph{total deterministic}, if there is exactly one
right-hand side for each possible left-hand side. %TODO "possible" necessary?
\item A transducer is \emph{linear}, if each variable that occurs on the
left-hand side of a rule occurs at most once in each right-hand side. 
\item A transducer is \emph{non-deleting}, if each variable that occurs on
the left-hand side of a rule occurs at least once in each right-hand side.
\item A transducer is single-state, or \emph{pure}, if $|Q| = 1$.
\end{compactitem}

By prepending $\mathbf{D_t}$, $\mathbf{L}$, $\mathbf{N}$, and $\mathbf{P}$
to $\mathbf{T}$ or $\mathbf{B}$, we denote the class of transductions
defined by TDFTT and BUFTT with the above constraints, respectively. For
example, $\mathbf{DLB}$ is the class of transductions definable by
deterministic linear BUFTT. 

In Subsection \ref{ssec_transducerdef}, we briefly mentioned that there
were transductions that could be defined by a BUFTT but not by a TDFTT, and
vice versa, i.e. that $\mathbf{T}$ and $\mathbf{B}$ are incomparable, that
is
$$\mathbf{T} \subsetneq \mathbf{B}$$
We further mentioned that the defining differences between BUFTT and TDFTT
were that BUFTT can process an input subtree nondeterministically, and then
copy the results using a later rule, alternatively discard the results
entirely.  TDFTT, by contrast, can copy an input subtree \emph{first}, and
then apply different states to the two outputs, or alternatively use the
same state, but do processing in the two copies with nondeterminstic
differences.

It would seem natural to use the restrictions detailed above to find a
''natural'' common subset of transductions to $\mathbf{T}$ and
$\mathbf{B}$. As it turns out, several such common subsets exist:

First of all, we note that by eliminating copying of subtrees (i.e.
imposing linearity), all that makes TDFTT more powerful than BUFTT is
eliminated, that is
$$\mathbf{LT} \subset \mathbf{LB} \subset \mathbf{B}$$

Further restricting the deletion of subtrees would seem to eliminate any
advantage to using BUFTT as well, leading us to state that
$$\mathbf{NLB} = \mathbf{NLT} \subset \mathbf{LT}$$

This equality relation is not immediately applicable in this thesis, but an
important subset of this class of transductions is, namely that of
\emph{finite state relabelings} ($\mathbf{QREL}$), where we further
restrict the right-hand sides of rules such that rules of TDFTT have the
form:
$$q[s[x_1,\ldots,x_k]] \rightarrow s'[q_1[x_1],\ldots,q_k[x_k]]$$ 
and of BUFTT have the form
$$s[q_1[x_1],\ldots,q_k[x_k]] \rightarrow q[s'[x_1,\ldots,x_k]$$

A different subset of both $\mathbf{B}$ and $\mathbf{T}$ can be obtained by
restricting the actual passing of (state) information, as long as no
non-deterministic copying is allowed. That is, 
$$\mathbf{PD_tB} = \mathbf{PD_tT}$$
Again, a formal proof is outside the scope of this thesis, but the equality
is well-known. This set is also known as the set of \emph{tree homomorphisms} 
($\mathbf{HOM}$).

While algorithms or properties could relatively easily be implemented in
Marbles to identify homomorphisms and finite state relabelings, this is not
currently available.

If $\mathbf{A}$ and $\mathbf{B}$ are classes of transductions, let
$\mathbf{A}\circ\mathbf{B}$ denote the class of transductions possible by
applying first a transduction from $\mathbf{A}$, and then applying one from
$\mathbf{B}$ to the result, that is we \emph{compose} a transduction in
$\mathbf{A}$ with one in $\mathbf{B}$. It is well-known that while string
transductions are closed under composition (that is, you gain no additional
possible string transductions by using two transducers rather than one),
the same is not true for tree transducers, that is
$$\mathbf{T} \subset \mathbf{T} \circ \mathbf{T}$$ and
$$\mathbf{B} \subset \mathbf{B} \circ \mathbf{B}$$

However, it is further known that every TDFTT can be represented by a
tree homomorphism followed by a linear TDFTT, both of which can be
represented by BUFTT. In addition, each BUFTT can be represented by a
finite state relabeling, followed by a tree homomorphism. That is,
$$\mathbf{T} \subset \mathbf{HOM} \circ \mathbf{LT} \subset \mathbf{B}
\circ \mathbf{B}$$ and
$$\mathbf{B} \subset \mathbf{QREL} \circ \mathbf{HOM} \subset \mathbf{T}
\circ \mathbf{T}$$

In the following sections, we will describe and implement algorithms that
can be used in constructive proofs of the last of the above statements.

\subsubsection{Bottom-up transducer splitting}

Recall that a BUFTT is a 5-tuple $B = (\Sigma, \Delta, Q, R, F)$, and
a TDFTT is a 5-tuple $T = (\Sigma, \Delta, Q, R, q_0)$,
as defined in Subsection \ref{ssec_transducerdef}. The idea of the BUFTT
decomposition algorithm is to define a ''transitional'' alphabet, which
is used by the finite state relabeling to store information indicating by
which tree-piece the node should be replace to produce the final output
tree. After that, a homomorphism is used to actually apply the
replacements. Formally,

\begin{itemize}
\item Let $B = (\Sigma, \Delta, Q, R, F)$ be the (input) BUFTT, then

\item $\Omega$ is the transitional alphabet, defined as follows: if 
$$s[q_1[x_1],\ldots,q_k[x_k]] \rightarrow q[t]$$ is a rule in $R$, then
$d_{t}$ is a (new) symbol in $\Omega_k$.

\item We define the TDFTT $T_1 = (\Sigma, \Omega, Q, R_1, F)$, a
finite state relabeling, with $R_1$ defined as follows: if 
$$s[q_1[x_1],\ldots,q_k[x_k]] \rightarrow q[t]$$ is a rule in $R$, then
$$q[s[x_1,\ldots,x_k]] \rightarrow d_{t}[q_1[x_1],\ldots,q_k[x_k]] \in R_1$$

\item We define the TDFTT $T_2 = (\Omega, \Delta, {q_s}, R_2, {q_s})$, a tree
homomorphism, with $R_2$ defined as follows: $$\forall d_t \in \Omega:
q_s[d_t[x_1,\ldots,x_k]] \rightarrow t' \in R_2$$ where $t'$ is the tree obtained by
replacing each element of $Q$ in $t$ by $q_s$, that is, the single state of
$T_2$.
\end{itemize}

Note that $\Omega$ is finite, as $R$ has a finite number of right-hand
sides. Further, looking at the rules in $R_1$, they all conform to the
pattern specified for TDFTT implementing finite state relabelings, meaning
that the transduction defined by $T_1$ is in $\mathbf{QREL}$. For $T_2$,
note that $Q$ is a singleton, and that we define exactly one rule in $R_2$
for every possible left-hand side, meaning that the transduction defined by
$T_2$ in in $\mathbf{PD_tT = HOM}$.

In lieu of a formal proof of $L(B) = L(T_1 \circ T_2$, we provide an
example transduction using the two methdods to illustrate how each step of
the transduction is preserved in the decomposition.

Consider the example BUFTT $B = (\Sigma, \Delta, Q, R, F)$ where
\begin{compactitem}
\item $\Sigma = {a_2, b_1, c_0}$ 
\item $\Delta = {f_2, g_1, h_1, i_0}$ 
\item $Q = {q_a, q_even, q_odd}$ 
\item $R = {$
\begin{compactitem}
\item $c \rightarrow q_even[i]$
\item $b[q_even[x_1]] \rightarrow q_odd[g[x_1]]$
\item $b[q_even[x_1]] \rightarrow q_odd[h[x_1]]$
\item $b[q_odd[x_1]] \rightarrow q_even[g[x_1]]$
\item $b[q_odd[x_1]] \rightarrow q_even[h[x_1]]$
\item $a[q_even[x_1],q_odd[x_2]] \rightarrow q_a[f[x_2,x_2]]$
\item $a[q_odd[x_1],q_even[x_2]] \rightarrow q_a[f[x_1,x_1]] }$
\end{compactitem}
\item $F = {q_a}$
\end{compactitem}

That is, each input subtree of $b[b[\ldots b[c]]]$ that consists of an odd
number of $b$s is copied (including the nondeterministic differences),
while a subtree with an even number of $b$s is deleted. We place heavy
constraints on what trees are accepted in order to make the rule set
smaller and simpler. Notably, we only accept input trees on the form
$a[b[b[\ldots b[c]]],b[b[\ldots b[c]]]]$.

As our standing example input in dealing with this transduction, we use the
simple, valid input tree $a[b[b[c]],b[c]]$, which gives the output tree set
${f[g[i],g[i]], f[h[i],h[i]]}$ as output, according to the rules governing
BUFTT.

Using the BUTSplitter algorithm to split $B$, we start by finding the
relevant intermediate alphabet 
$$\Omega={d_{i}_0, d_{g[x_1]}_1, d_{h[x_1]}_1, d_{f[x_2,x_2]}_1, d_{f[x_1,x_1]}_1}$$,

With this done, we move on to the first TDFTT $T_1$ (the finite state
relabeling), where

\begin{compactitem}
\item $\Sigma = {a_2, b_1, c_0}$,
\item $\Delta = \Omega = {d_{i}_0, d_{g[x_1]}_1, d_{h[x_1]}_1, d_{f[x_2,x_2]}_2, d_{f[x_1,x_1]}_2}$,
\item $Q = {q_a, q_{even}, q_{odd}}$ 
\item $R = {$
\begin{compactitem}
\item $q_{even}[c] \rightarrow d_{i}$
\item $q_{even}[b[x_1]] \rightarrow d_{g[x_1]}[q_{odd}[x_1]]$
\item $q_{even}[b[x_1]] \rightarrow d_{h[x_1]}[q_{odd}[x_1]]$
\item $q_{odd}[b[x_1]] \rightarrow d_{g[x_1]}[q_{even}[x_1]]$
\item $q_{odd}[b[x_1]] \rightarrow d_{h[x_1]}[q_{even}[x_1]]$
\item $q_a[a[x_1,x_2]] \rightarrow d_{f[x_1,x_1]}[q_{odd}[x_1],q_{even}[x_2]]$
\item $q_a[a[x_1,x_2]] \rightarrow d_{f[x_2,x_2]}[q_{even}[x_1],q_{odd}[x_2]] }$
\end{compactitem}
\item $q_0 = {q_a}$
\end{compactitem}

Note that even though only one of the two subtrees of the $d_{f[\ldots]}$
is used in the final output, we still need to do computations on both
subtrees to determine that their heights are correct (even and odd,
respectively). Any rejection of an input tree happens in this transducer,
through failing to apply rules to a subtree.

The homomorphism $T_2$ that completes the tree transduction is comparatively
simple; in TDFTT form:

\begin{compactitem}
\item $\Sigma = \Omega = {d_{i}_0, d_{g[x_1]}_1, d_{h[x_1]}_1, d_{f[x_2,x_2]}_2, d_{f[x_1,x_1]}_2}$,
\item $\Delta = {f_2, g_1, h_1, i_0}$ 
\item $Q = {q_{only}}$
\item $R = {$
\begin{compactitem}
\item $q_{only}[d_{i}] \rightarrow i$
\item $q_{only}[d_{g[x_1]}[x_1]] \rightarrow g[q_{only}[x_1]]$
\item $q_{only}[d_{h[x_1]}[x_1]] \rightarrow h[q_{only}[x_1]]$
\item $q_{only}[d_{f[x_1,x_1}[x_1,x_2]] \rightarrow f[q_{only}[x_1],q_{only}[x_1]]$
\item $q_{only}[d_{f[x_2,x_2}[x_1,x_2]] \rightarrow f[q_{only}[x_2],q_{only}[x_2]] }$
\end{compactitem}
\item $q_0 = q_{only}$
\end{compactitem}

There is nothing inherently interesting about this transduction, save that
it applies the copying and deletion that is specified by the relabeling
(the $d_{f[\ldots]}$ rules, again). 

By applying only the relabeling $T_1$ to the input tree $a[b[b[c]],b[c]]$,
we arrive at the intermediate tree set
$$
{ &d_{f[x_2,x_2]}[d_{h[x_1]}[d_{h[x_1]}[i]],d_{g[x_1]}[i]],\\
&d_{f[x_2,x_2]}[d_{h[x_1]}[d_{h[x_1]}[i]],d_{h[x_1]}[i]],\\
&d_{f[x_2,x_2]}[d_{g[x_1]}[d_{h[x_1]}[i]],d_{h[x_1]}[i]],\\
&d_{f[x_2,x_2]}[d_{h[x_1]}[d_{g[x_1]}[i]],d_{h[x_1]}[i]],\\
&d_{f[x_2,x_2]}[d_{g[x_1]}[d_{g[x_1]}[i]],d_{h[x_1]}[i]],\\
&d_{f[x_2,x_2]}[d_{g[x_1]}[d_{h[x_1]}[i]],d_{g[x_1]}[i]],\\
&d_{f[x_2,x_2]}[d_{g[x_1]}[d_{g[x_1]}[i]],d_{g[x_1]}[i]],\\
&d_{f[x_2,x_2]}[d_{h[x_1]}[d_{g[x_1]}[i]],d_{g[x_1]}[i]] }
$$

Obviously, all derivations starting with the rule
$$q_a[a[x_1,x_2]] \rightarrow d_{f[x_1,x_1]}[q_{odd}[x_1],q_{even}[x_2]]$$
will fail, as it is not possible to construct a successful run from
$q_{odd}$ on the subtree $b[b[c]]$, and neither from $q_{even}$ on $b[c]$.
Nevertheless, the amount of trees would look strange, given that the
correct output according to $B$ is a set of only two trees. However, the
homomorphism will delete the larger subtrees (which are all that differs
within two groups of four trees each), leaving a set of two trees.



%As for the equality $L(B) = L(T_1 \circ T_2)$, we sketch a proof by
%induction on the input tree $t$:
%
%\begin{itemize}
%\item First, note that due to the initial states of $T_1$ being $F$, it is
%suffient to prove that if there is a derivation by $B$ on a subtree $t$
%that can end up in $q[t']$, with $t' \in T_\Delta$, then $T_1 \circ T_2$,
%with $T_1$ starting in state $q$, will end up producing the same $t'$.
%
%\item Consider the case where $t = s$, that is, the input is a single leaf.
%The result from $B$, started on $t$, is then the set $qts = {q[t']: s \rightarrow
% q[t'] \in R}$. For each such right-hand side we expect that $T_1 \circ T_2$,
%started in $q$, and run on $s$ will produce ${t' : q[t'] \in qts}$. By
%looking at the definitions of $\Omega$, $R_1$ and $R_2$, we see that $T_1$,
%ran on $q[s]$, will output the set of trees $dts = {d_t : q[s] \rightarrow d_t
%  \in R_1}$, further, $T_2$ will transform each of these trees to the
%''embedded'' set of trees ${t : d_t \in dts}$. By the definitions of $R_1$,
%$R_2$ and $\Omega$, this is exactly the set ${t' : q[t'] \in qts}$.
%
%\item Moving on to the case of $t = s[t_1,\ldots,t_k]$, we know that $B$,
%started on $t$ will produce a set $qts = {q[t']: s[x_1,+ldots,x_k]
%  \rightarrow q[t'] \in R}$. For
%each such right-hand side
%
%\end{itemize}



\subsubsection{Top-down transducer splitting}


