\subsection{Bottom-up transducer splitting}

Recall that a BUFTT is a 5-tuple $B = (\Sigma, \Delta, Q, R, F)$, and
a TDFTT is a 5-tuple $T = (\Sigma, \Delta, Q, R, q_0)$,
as defined in Subsection \ref{ssec_transducerdef}. Further, according to
the above reasoning, the features of BUFTT that can not be realised using a
single TDFTT is that of deleting subtrees based on the features of that
subtree, and that of copying an already processed output tree into multiple
copies.

The idea of the BUFTT decomposition algorithm is to define a
''transitional'' alphabet, which is used by the finite state relabeling to
store information indicating by which tree-piece the node should be replace
to produce the final output tree. After that, a homomorphism is used to
actually apply the replacements. Formally,

\begin{itemize}
\item Let $B = (\Sigma, \Delta, Q, R, F)$ be the (input) BUFTT, then

\item $\Omega$ is the transitional alphabet, defined as follows: if 
$$s[q_1[x_1],\ldots,q_k[x_k]] \rightarrow q[t]$$ is a rule in $R$, then
$d_{t}$ is a (new) symbol in $\Omega_k$.

\item We define the TDFTT $T_1 = (\Sigma, \Omega, Q, R_1, F)$, a
finite state relabeling, with $R_1$ defined as follows: if 
$$s[q_1[x_1],\ldots,q_k[x_k]] \rightarrow q[t]$$ is a rule in $R$, then
$$q[s[x_1,\ldots,x_k]] \rightarrow d_{t}[q_1[x_1],\ldots,q_k[x_k]] \in R_1$$

\item We define the TDFTT $T_2 = (\Omega, \Delta, {q_{only}}, R_2, {q_{}only})$, a tree
homomorphism, with $R_2$ defined as follows: $$\forall d_t \in \Omega:
q_{only}[d_t[x_1,\ldots,x_k]] \rightarrow t' \in R_2$$ where $t'$ is the tree obtained by
replacing each element of $Q$ in $t$ by $q_{only}$.
\end{itemize}

Note that $\Omega$ is finite, as $R$ has a finite number of right-hand
sides. Further, looking at the rules in $R_1$, they all conform to the
pattern specified for TDFTT implementing finite state relabelings, meaning
that the transduction defined by $T_1$ is in $\mathbf{QREL}$. For $T_2$,
note that $Q$ is a singleton, and that we define exactly one rule in $R_2$
for every possible left-hand side, meaning that the transduction defined by
$T_2$ in in $\mathbf{PD_tT = HOM}$.

In lieu of a formal proof of $L(B) = L(T_1 \circ T_2$, we provide an
example transduction using the two methdods to illustrate how each step of
the transduction is preserved in the decomposition.

Consider the example BUFTT $B = (\Sigma, \Delta, Q, R, F)$ where
\begin{compactitem}
\item $\Sigma = {a_2, b_1, c_0}$ 
\item $\Delta = {f_2, g_1, h_1, i_0}$ 
\item $Q = {q_a, q_even, q_odd}$ 
\item $R = {$
\begin{compactitem}
\item $c \rightarrow q_even[i]$
\item $b[q_even[x_1]] \rightarrow q_odd[g[x_1]]$
\item $b[q_even[x_1]] \rightarrow q_odd[h[x_1]]$
\item $b[q_odd[x_1]] \rightarrow q_even[g[x_1]]$
\item $b[q_odd[x_1]] \rightarrow q_even[h[x_1]]$
\item $a[q_even[x_1],q_odd[x_2]] \rightarrow q_a[f[x_2,x_2]]$
\item $a[q_odd[x_1],q_even[x_2]] \rightarrow q_a[f[x_1,x_1]] }$
\end{compactitem}
\item $F = {q_a}$
\end{compactitem}

That is, each input subtree of $b[b[\ldots b[c]]]$ that consists of an odd
number of $b$s is copied (including the nondeterministic differences),
while a subtree with an even number of $b$s is deleted. We place heavy
constraints on what trees are accepted in order to make the rule set
smaller and simpler. Notably, we only accept input trees on the form
$a[b[b[\ldots b[c]]],b[b[\ldots b[c]]]]$.

As our standing example input in dealing with this transduction, we use the
simple, valid input tree $a[b[b[c]],b[c]]$, which gives the output tree set
${f[g[i],g[i]], f[h[i],h[i]]}$ as output, according to the rules governing
BUFTT.

Using the BUTSplitter algorithm to split $B$, we start by finding the
relevant intermediate alphabet 
$$\Omega={d_{i}_0, d_{g[x_1]}_1, d_{h[x_1]}_1, d_{f[x_2,x_2]}_1, d_{f[x_1,x_1]}_1}$$,

With this done, we move on to the first TDFTT $T_1$ (the finite state
relabeling), where

\begin{compactitem}
\item $\Sigma = {a_2, b_1, c_0}$,
\item $\Delta = \Omega = {d_{i}_0, d_{g[x_1]}_1, d_{h[x_1]}_1, d_{f[x_2,x_2]}_2, d_{f[x_1,x_1]}_2}$,
\item $Q = {q_a, q_{even}, q_{odd}}$ 
\item $R = {$
\begin{compactitem}
\item $q_{even}[c] \rightarrow d_{i}$
\item $q_{even}[b[x_1]] \rightarrow d_{g[x_1]}[q_{odd}[x_1]]$
\item $q_{even}[b[x_1]] \rightarrow d_{h[x_1]}[q_{odd}[x_1]]$
\item $q_{odd}[b[x_1]] \rightarrow d_{g[x_1]}[q_{even}[x_1]]$
\item $q_{odd}[b[x_1]] \rightarrow d_{h[x_1]}[q_{even}[x_1]]$
\item $q_a[a[x_1,x_2]] \rightarrow d_{f[x_1,x_1]}[q_{odd}[x_1],q_{even}[x_2]]$
\item $q_a[a[x_1,x_2]] \rightarrow d_{f[x_2,x_2]}[q_{even}[x_1],q_{odd}[x_2]] }$
\end{compactitem}
\item $q_0 = {q_a}$
\end{compactitem}

Note that even though only one of the two subtrees of the $d_{f[\ldots]}$
is used in the final output, we still need to do computations on both
subtrees to determine that their heights are correct (even and odd,
respectively). Any rejection of an input tree happens in this transducer,
through failing to apply rules to a subtree.

The homomorphism $T_2$ that completes the tree transduction is comparatively
simple; in TDFTT form:

\begin{compactitem}
\item $\Sigma = \Omega = {d_{i}_0, d_{g[x_1]}_1, d_{h[x_1]}_1, d_{f[x_2,x_2]}_2, d_{f[x_1,x_1]}_2}$,
\item $\Delta = {f_2, g_1, h_1, i_0}$ 
\item $Q = {q_{only}}$
\item $R = {$
\begin{compactitem}
\item $q_{only}[d_{i}] \rightarrow i$
\item $q_{only}[d_{g[x_1]}[x_1]] \rightarrow g[q_{only}[x_1]]$
\item $q_{only}[d_{h[x_1]}[x_1]] \rightarrow h[q_{only}[x_1]]$
\item $q_{only}[d_{f[x_1,x_1}[x_1,x_2]] \rightarrow f[q_{only}[x_1],q_{only}[x_1]]$
\item $q_{only}[d_{f[x_2,x_2}[x_1,x_2]] \rightarrow f[q_{only}[x_2],q_{only}[x_2]] }$
\end{compactitem}
\item $q_0 = q_{only}$
\end{compactitem}

The rule set of this homomorphism should be sufficient to demonstrate how
deletion and copying is translated in the BUTSplitter algorithm.
Specifically, the $d_{f[\ldots]}$ rules are again the rules that are of
interest.

By applying only the relabeling $T_1$ to the input tree $a[b[b[c]],b[c]]$,
we arrive at the intermediate tree set
$$
{ 
&d_{f[x_2,x_2]}[d_{h[x_1]}[d_{h[x_1]}[i]],d_{h[x_1]}[i]],\\
&d_{f[x_2,x_2]}[d_{h[x_1]}[d_{h[x_1]}[i]],d_{g[x_1]}[i]],\\
&d_{f[x_2,x_2]}[d_{h[x_1]}[d_{g[x_1]}[i]],d_{h[x_1]}[i]],\\
&d_{f[x_2,x_2]}[d_{h[x_1]}[d_{g[x_1]}[i]],d_{g[x_1]}[i]],\\
\\
&d_{f[x_2,x_2]}[d_{g[x_1]}[d_{h[x_1]}[i]],d_{h[x_1]}[i]],\\
&d_{f[x_2,x_2]}[d_{g[x_1]}[d_{h[x_1]}[i]],d_{g[x_1]}[i]],\\
&d_{f[x_2,x_2]}[d_{g[x_1]}[d_{g[x_1]}[i]],d_{h[x_1]}[i]],\\
&d_{f[x_2,x_2]}[d_{g[x_1]}[d_{g[x_1]}[i]],d_{g[x_1]}[i]]
}
$$

Obviously, all derivations starting with the rule
$$q_a[a[x_1,x_2]] \rightarrow d_{f[x_1,x_1]}[q_{odd}[x_1],q_{even}[x_2]]$$
will fail, as it is not possible to construct a successful run from
$q_{odd}$ on the subtree $b[b[c]]$, and neither from $q_{even}$ on $b[c]$.
Nevertheless, the amount of trees would look strange, given that the
correct output according to $B$ is a set of only two trees. However, the
homomorphism will delete the larger subtrees (which are all that differs
within two groups of four trees each), leaving the correct set of two
trees:

$${f[g[i],g[i]], f[h[i],h[i]]}a$$

\subsection{Top-down transducer splitting}

In the case of TDFTT, the capabilities that a single BUFTT is unable to
reproduce is, as stated, to copy an input subtree and use different
processing for different copies, either through starting in different
states or through nondeterminism having different outcomes.

We again define a transitional alphabet $\Omega$ to realise the goal.
However, instead of storing information about the nondeterminism, copying
and deletion applied in the labels, the first BUFTT is a homomorphism,
''exploding'' the input tree into a much ''wider'' version of the input
tree. Specifically, each subtree $b[t]$ is converted into a new subtree
$b[t,t,\ldots,t]$, where the number of $t$'s in the output is dependent on
the maximum number of copies that might be made of a subtree in the input
TDFTT. After copying is done, a linear BUFTT does the computations required
on all copies simultaneously, after which the unneeded subtrees are simply
deleted. Formally:

\begin{itemize}
\item Let $T = (\Sigma, \Delta, Q, R, q_0)$ be the input TDFTT, then

\item $\Omega$ is the transitional alphabet, defined as follows: if $n$ is
the maximum number of copies made by any rule in $R$ of any subtree, and 
$s_k \in \Sigma$, then $s_{k*n} \in \Omega$.

\item We define a homomorphism, $B_1 = (\Sigma, \Omega, q_{only}, R_1, q_{only})$
where $R_1$ is defined as follows: $\forall s_k \in \Sigma:
s[q_{only}[x_1],\ldots,q_{only}[x_k]] \rightarrow
\q_{only}[s[x_1,x_1,\ldots,x_k,x_k]] \in R_1$, where each $x_i$ occurs
exactly $n$ times in the right-hand side of the rule.

\item We define a linear BUFTT, $B_2' = (\Omega, \Delta, Q, R_2, q_0)$,
	where $R_2$ is discussed further below.
\end{itemize}

$R_2$ is slightly more complex than $R_1$ in the bottom-up case. Suppose
that $$q[s[x_1,\ldots,x_k]] \rightarrow c[q_1[x_i_1],\ldots,[q_n[x_i_n]]]$$
is a rule in $R$, then a set of rules exists in $R_2$, where 
%TODO: R_2

However, the BUFTT $B_2'$ is not quite sufficient to define the translation
we seek, as irrelevant subtrees, which are never even considered by $T$
will be interpreted by $B_2'$ as faulty (i.e. there are no rules to apply
at a certain place), halting the transduction. A simple solution for this
problem is to make $B_2'$ \emph{total}, adding a \emph{dead state} to $Q$,
and adding rules to $R_2$ such that all possible previously undefined
transitions go to the new dead state.

