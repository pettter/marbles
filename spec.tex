\documentclass[a4paper,10pt]{article}
\usepackage[english]{babel,varioref}
\usepackage[T1]{fontenc}
\usepackage{epsfig,fancyhdr,verbatim}
\pagenumbering{arabic}

\section{Background}

The theory of formal tree languages is a rapidly evolving field of
research, with applications ranging from microcode optimisation in
compilers to translation of natural languages. The basis of the field is
the theory of formal languages as made popular by Chomsky, but instead of
working on strings of symbols, the data is structured as \emph{trees},
where a tree is defined as a root symbol with a (finite) number of
\emph{subtrees}.

This class of structure turns out to be a reasonable way of representing
many types of structured and semistructured data, as shown for example by
the wide use of XML for data transfer applications. XML files repesent what
is called \emph{unranked} trees. That is to say, any node may have any
number of subtrees, regardless of the symbol at that node. This is
contrasted by the \emph{ranked} trees, where each symbol in the alphabet
has a \emph{rank} denoting the number of subtrees of every tree rooted by
that symbol.

\section{Motivation}

As mentioned above, research into various tree language and tree transducer
formalisms and applications is a very active field. However, development of
proof-of-concept code and applications of implemented algorithms to related
formalisms are hampered by the lack of a general framework for working with
trees and tree formalisms. Many specific programming tools exist, for many
different domains, but are often overly focused on a specific domain or
formalism, excluding potential research into related areas.

Hence, the need for a generalised programming framework for working with
trees and various tree formalisms in a simple and straightforward manner.
In \cite{marbles}, the basic design ideas for such a framework is proposed,
and the reasonings in this and the previous sections are stated in an
extended and much clearer manner.

\section{Specification}

The goal of the thesis is to produce a prototype of the framework, not a
complete implementation. Even so, an investigation into various classes of
tree formalisms to be represented should be made, so as to have the
information necessary to make reasonable choices for various implementation
details.

That is, the project can be divided into two conceptually distinct
activities; investigating the current state of research into tree
automata/tree formalisms of various sorts, and implementing a prototype of
the Marbles system with regard to the information gathered.

In particular, the prototype should
\begin{itemize}
\item work, as in be usable for at least one task that the full system
should cover,
\item include a concrete formulation of the concepts described in
\cite{marbles}, focusing on the implementation of same, and 
\item compose these into a reasonable architecture for the complete system,
	although it is likely this will serve more as a basis for discussion
	than as a foundation for the final version of the framework.
\end{itemize}

